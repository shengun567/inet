\section{Creating and Running Simulations}

\subsection{Setting up a simulation}

The INET Framework builds upon OMNeT++, and uses the same concept: modules
that communicate by message passing. Hosts, routers, switches and other
network devices are represented by OMNeT++ compound modules.

To create a simulation, you would write a NED file that contains the network,
i.e. routers, hosts and other network devices connected together. You can
use a text editor or the IDE's graphical editor to create the network.

Modules in the network contain a lot of unassigned parameters, which need
to be assigned before the simulation can be run.
The name of the network to be simulated, parameter values
and other configuration option need to be specified in the \ttt{omnetpp.ini}
file.

\ttt{omnetpp.ini} contains parameter assignments as \textit{key=value}
lines, where each key is a wildcard pattern. The simulator matches these
wildcard patterns against full path of the parameter in the module tree
(something like \ttt{Test.host[2].tcp.nagleEnabled}), and value from
the first match will be assigned for the parameter. If no matching line is
found, the default value in the NED file will be used. (If there is no
default value either, the value will be interactively prompted for, or, in
case of a batch run, an error will be raised.)

There are two kinds of wildcards: a single asterisk \ttt{*} matches at most
one component name in the path string, while double asterisk \ttt{**} may
match multiple components. Technically: \ttt{*} never matches a dot or a
square bracket (\ttt{.}, \ttt{[}, \ttt{]}), while \ttt{**} can match any of
them. Patterns are also capable of expressing index ranges
(\ttt{**.host[1..3,5,8].tcp.nagleEnabled}) and ranges of numbers embedded
in names (\ttt{**.switch\{2..3\}.relayUnitType}).

OMNeT++ allows several configurations to be put into the \ttt{omnetpp.ini}
file under \ttt{[Config <name>]} section headers, and the right
configuration can be selected via command-line options when the simulation
is run. Configurations can also build on each other: \ttt{extends=<name>}
lines can be used to set up an inheritance tree among them. This feature
allows minimizing clutter in ini files by letting you factor out common
parts. (Another ways of factoring out common parts are ini file inclusion
and specifying multiple ini files to a simulation.) Settings in the
\ttt{[General]} section apply to all configurations, i.e. \ttt{[General]}
is the root of the section inheritance tree.

Parameter studies can be defined by specifying multiple values for a
parameter, with the \ttt{\$\{10,50,100..500 step 100, 1000\}} syntax;
a repeat count can also be specified.


\subsection{Running simulations}

TODO how to run

\subsection{Result Collection and Analysis}

how to configure result collection

how to analyze results



\subsection{Setting up a wired network simulation}

For an introduction, in this section we show you how to set up simulations
of wired networks using PPP or Ethernet links with autoconfigured static IP
routing. (If your simulation involves more, such as manually configured
routing tables, dynamic routing, MPLS, or IPv6 or other features and protocols,
you'll find info about them in later chapters.)

Such a network can be assembled using the predefined \nedtype{StandardHost}
and \nedtype{Router} modules. For automatic IP address assignment and
static IP routing we can use the \nedtype{Ipv4NetworkConfigurator} utility
module.

TODO:  ethg, pppg;  automatically expand (++)

todo which modules are needed into it, what they do, etc.

how to add apps, etc


\subsection{Setting up a wireless network simulation}

In this section we show you how to set up wireless network simulations using
IEEE 802.11 links with autoconfigured static IP routing. Such a network can be
assembled using the predefined \nedtype{WirelessHost} and \nedtype{AccessPoint}
modules. \nedtype{RadioMedium} is also required for wireless simulations. It
keeps track of which nodes are within interference distance of other nodes.


